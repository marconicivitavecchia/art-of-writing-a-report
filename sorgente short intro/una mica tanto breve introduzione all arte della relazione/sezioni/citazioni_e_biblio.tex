						\section{Citazioni, note e bibliografia}

Scrivere un documento significa anche attingere ad altri tipi di informazioni, studi precedenti come anche a esperienze di altro genere. Potrebbe quindi nascere la necessità di doversi riferire a uno o più articoli, report, analisi, studi o norme che possono essere citati più volte e in più modi.

In questi casi il testo può essere riportato sia nel \hlightit{corpo del testo} sia \hlightit{fuori testo}. Si adotta il primo metodo nel caso in cui il testo da citare sia relativamente breve, il secondo modo se la citazione è ben più complessa ed estesa.

Nel caso di una citazione nel corpo, il testo dovrà essere messo in evidenza (solitamente si usa lo stile \textit{corsivo}) e sarà delimitato da virgolette (i cosiddetti \hlightit{caporali} 
<<\ldots~testo~\ldots>>) come l'esempio che segue:

\begin{quote}\small
[\ldots]~Lo studio delle leggi che formano il nucleo della meccanica classica e di quelle sulla gravitazione, non può che ricondursi a Isaac Newton che, proprio sulla forza di gravità e della luna scrisse: <<\emph{Cominciai a pensare che la forza di gravità potesse estendersi all'orbita della Luna}>>[\ldots]
\end{quote}

Nel caso di citazioni fuori dal corpo, il testo viene scritto su una paragrafo a parte. Per dare maggiore enfasi alla citazione e per non rischiare di confonderla con il testo che la ospita, si utilizza un \hlightit{paragrafo rientrato} e un testo con un corpo più piccolo:

\begin{quote}\small
[\ldots]~il calcolo delle correnti di cortocircuito nelle reti alimentate a bassa tensione, deve tener conto di quanto stabilito dalla norma \textsc{cei 11-28} (\textit{Guida d'applicazione per il calcolo delle correnti di cortocircuito nelle reti radiali a bassa tensione}) che definisce il corto circuito come:
%%
		\vspace{-5pt}\begin{changemargin}{5mm}{5mm}\footnotesize
      Connessione accidentale o intenzionale, di resistenza o impedenza relativamente bassa, di due o più punti in un circuito che normalmente sono a tensione diversa (IEV, Pubblicazione IEC 50 151-03-41 Chapter 151: Electrical and magnetic devices).
		\end{changemargin}
\end{quote}\vspace{-5pt}

Si noti l'uso di un paragrafo separato e il corpo del font più piccolo.

Un aspetto di fondamentale e deontologica correttezza e importanza è, nel caso si faccia uso di citazioni, di studi o analisi provenienti da altri, come anche parti di testo prese integralmente da articoli o altro, è l'indicazione delle fonti.

Le frazioni di testo prese da supporti cartacei o digitali o provenienti dal web o da altri metodi di trasmissione, anche se sommariamente rielaborate con il ricorso di sinonimi, devono essere accompagnate da note in cui viene specificato l'autore o gli autori, i lavori a cui ci si è ispirati, il titolo dell'opera, l'\textsc{url} dell'articolo eccetera. In caso contrario si commette un abuso che non consiste solo nel millantare idee non proprie, ma di consumare un reato accademicamente ben più grave: il \hlightit{plagio}.

L'Università di Pisa definisce il plagio nel seguente modo:

                  \begin{quote}
										[il plagio] è definito come la parziale o totale attribuzione di idee, ricerche o scoperte altri a se stessi o ad un altro autore, a prescindere dalla lingua in cui queste sono ufficialmente presentate o divulgate, o nell'ommissione della citazione delle fonti.
                  \end{quote}

L'indicazione della fonte può seguire la citazione o essere specificata in una nota a piè di pagina o a margine del testo. A tali modi si preferisce, se possibile e specie se il numero delle citazioni o dei riferimenti a altri lavori diventa importante, l'usi della \hlightit{bibliografia} inserita in una sezione a parte e a fine dell'elaborato.

I modi con cui citare un particolare elaborato o lavoro sono molti e dipendono anche dalla natura del testo che si sta scrivendo. (umanistico, tecnico-scientifico, filosofico, giuridico \ecc).

Una citazione bibliografica è composta dai seguenti elementi:

\begin{quotation}\small
 [<etichetta>] <autore o autori (in tondo)>. <\textit{Titolo} in \textit{corsivo}>. <Editore>, <anno di pubblicazione>, <codice \textsc{isbn}> (<pagina in cui compare la citazione>)
\end{quotation}

Per esempio:
%%
		\begin{quote}\small
			\begin{enumerate}[leftmargin=20mm]
				\item[$\textlbrackdbl$HS16$\textrbrackdbl$] M.~Hughes e B.~Sansone. \textit{Arduino per tecnici, ingegneri e maker}. Tecniche nuove, 2016. \textsc{isbn}: 9788848131780 (cit. a p.~4)
			\end{enumerate}
		\end{quote}

Se la citazione riguarda un'opera o un articolo disponibile su \textsc{web}, la citazione potrà essere scritta come:
%%
		\begin{quote}\small
			\begin{enumerate}[leftmargin=25mm]
				\item[$\textlbrackdbl$avrdude$\textrbrackdbl$] Systutorial. \textit{Linux Man Pages--averdude man page}. \url{https://www.systutorials.com/docs/linux/man/1-avrdude} (cit. a p.~13)
			\end{enumerate}
		\end{quote}
%%
e nel caso sia poi possibile scaricare un documento in formato elettronico:
%%
		\begin{quote}\small
			\begin{enumerate}[leftmargin=25mm]
				\item[$\textlbrackdbl$Plan$\textrbrackdbl$] PlatformIO. \textit{PlatformIO Documentation}. jun 25,2018. \url{https://media.readthedocs.org/pdf/platformio/latest/platformio.pdf} (cit. a p.~6)
			\end{enumerate}
		\end{quote}
%%
dove [HS16], [avrdude], [Plan] sono possibili etichette (in gergo \hlightit{label}) che compaiono nei punti in cui sono presenti le citazioni. Le label possono essere non solo alfabetiche, ma anche unicamente numeriche o alfanumeriche.


					\section{Conclusioni}

Scrivere una buona relazione, fosse anche un breve report, un articolo o saggio breve comporta inevitabilmente una buona, anzi ottima, conoscenza della materia o dei singoli argomenti che si stanno trattando (nei lavori di gruppo, i componenti potrebbero infatti essere esperti o competenti solo per una frazione del lavoro che si sta svolgendo).

Nel caso in cui il documento sia strutturato e composto da più di quindici-venti pagine contenenti grafici e tabelle, l'autore dovrà considerare la possibilità di inserire nelle prime pagine, un indice generale che elenca divisi per sezioni o eventualmente anche per sottosezione gli argomenti trattati e un indice delle figure e delle tabelle. Le pagine dovranno essere ovviamente numerate.

Si raccomanda anche di comporre in modo corretto la cosiddetta \hlightit{prima pagina di copertina} che dovrà comprendere nome e cognome dell'autore o degli autori, l'istituzione, azienda o altro di cui si fa parte, il titolo del documento, la data e, se possibile, un \hlightit{logo}.

Possibili \textit{prime di copertina} sono quelle indicate in figura~\ref{fig:fpage} a pagina~\pageref{fig:fpage}.

Una relazione dovrà essere strutturata e comprendere:
%%
\begin{enumerate}
 \item oggetto e scopo;
 %%
 \item schemi, disegni e tabelle;
 %%
 \item nel caso si faccia un uso esteso di una particolare simbologia o di acronimi, inserire in apertura una tabella dei simboli e una lista degli acronimi comprensivi di breve significato;
 %%
 \item eventuali definizioni date a termini tecnici;
 %%
 \item esposizione della teoria, delle ipotesi e dei modelli matematici o circuitali usati nel corso delle prove, misure, esperimento \ecc;
 %%
 \item calcoli preliminari e scelta degli strumenti le cui caratteristiche devono essere oggetto di una tabella a parte;
 %%
 \item condotta della prova, misure e, se richiesti o necessari, raccolta di altri dati ritenuti necessari (per esempio temperatura ambientale, tasso di umidità, \ecc);
 %%
 \item analisi dei dati e dei valori misurati, eventuale esclusione di dati ritenuti errati o incongruenti. La scelta dell'esclusione va in ogni caso motivata;
 %%
 \item definizione e spiegazione dei termini che compaiono nelle formule e calcoli. Nel caso in cui i calcoli siano numerosi e prodotti da un numero ristretto di formule continuamente reiterate, anziché ripeterli, si descriverà una sola volta il significato delle formule e dei termini che vi compaiono e si inseriranno i valori elaborati a parte, in una o più tabelle;
 %%
 \item analisi dei valori calcolati;
 %%
 \item conclusioni e deduzioni.
\end{enumerate}

\begin{landscape}
\begin{figure}[htp]
	\centering
%	\subfloat[Prima Pagina di copertina.]{%
		\includegraphics[width=0.65\textheight]{figure/frontpage_1.pdf}%
%	}
\hfill
%	\subfloat[<caption_fig2>]{%
		\includegraphics[width=0.65\textheight]{figure/frontpage_2.pdf}%
%	}
	\caption{Possibili prime pagine di copertina.}
	\label{fig:fpage}
\end{figure}
\end{landscape}






























