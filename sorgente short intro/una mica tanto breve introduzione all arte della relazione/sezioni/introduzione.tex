Con il termine \textit{relazione} si indica un rapporto scritto tecnico-scientifico (ma non solo), individuale o collegiale, che consiste nell'esposizione di dati, risultati e conclusioni relativi a un esperimento, una perizia o a uno studio di carattere sociale, economico, commerciale, scientifico, strategico eccetera.

La relazione è quindi un documento, dove precisione, dettaglio, oggettività, esposizione, conoscenze, citazioni e collegamenti vengono utilizzati per descrivere un particolare fenomeno la cui osservazione, sperimentazione, correlazione dei dati, definizione di ipotesi, elaborazione di un modello (per esempio matematico, fisico, economico, sociale \ecc), formalizzazione o dimostrazione di una teoria e infine una conclusione, consente non solo di conoscere e comprendere il fenomeno osservato, ma anche mostrare le modalità con cui sono stati acquisiti e gestiti i dati.

La relazione ha anche l'importante scopo di permettere a chi legge di ripetere quello stesso esperimento in modo da ricavare i dati necessari --gli stessi o diversi-- per confermare o confutare teorie, ipotesi, modelli e metodi.

La relazione, intesa quindi come fonte di dati, informazioni o elementi, è ben più articolata e complessa di una semplice testimonianza, di un messaggio, di una disposizione o di una comunicazione.

La relazione --quella tecnico-scientifica in particolare-- consiste anche nell'organizzazione di una modalità di ricerca che consenta la corretta analisi e interpretazione delle informazioni. Modalità che comprende strumenti per l'acquisizione dei dati, di un sistema con cui classificarli (per esempio scartare quelli ritenuti errati) e di un modello (tipicamente matematico--statistico) con cui analizzarli.

Una delle caratteristiche imprescindibili di una relazione è quella di possedere un punto di vista assolutamente oggettivo e impersonale, tale da evitare non solo parzialità di ogni genere, ma anche lo spostamento dell'attenzione da ciò che si sta osservando e analizzando.

L'indagine del fenomeno studiato, che è poi il vero oggetto di una relazione, pur oggettiva per definizione, rappresenta inevitabilmente l'autore (o gli autori) e riflette, per questo, il suo pensiero analitico, la solidità delle sue conoscenze e i meccanismi che portano al ragionamento e alla deduzione.



