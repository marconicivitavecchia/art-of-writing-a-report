%%%%%%%%%%%%%%%%%%%%%%%%%%%%%%%%%%%%%%%%%%%%%%%%%%%%
%% Comandi per indicare resistenze a stella e a triangolo
%% Uso \Rt o \Rs senza argomento produce una "R" con pedice stella
%% o triangolo; \Rt[n] o \Ry[n] come sopra ma con indicazione
%% numerica o alfanumerica eccetera
\newcommand{\Ry}[1][]{R_{#1\text{%
																	\lower-1.47ex\hbox{\rotatebox{180}{\ttfamily Y}}}}}
\newcommand{\Rt}[1][]{R_{#1\bigtriangleup}}
\newcommand{\Cy}{C_{\text{\lower-1.47ex\hbox{\rotatebox{180}{\ttfamily Y}}}}}
\newcommand{\Zy}[1][]{Z_{#1\text{%
																	\lower-1.47ex\hbox{\rotatebox{180}{\ttfamily Y}}}}}
\newcommand{\Zt}[1][]{Z_{#1\bigtriangleup}}
\newcommand{\Tt}{T_{\bigtriangleup}}
\newcommand{\jXc}{\jmath X_C}
\newcommand{\jXl}{\jmath X_L}

\newcommand{\eass}[1][]{\delta_{a#1}}
\newcommand{\erel}[1][]{\varepsilon_{r#1}}


%%%%%%%%%%%%%%%%%%%%%%%%%%%%%%%%%%%%%%%%%%%%%%%%%%%%%
%% Unità di misura da usare direttamente
%% \volt
%% \mV
%% \ohm
%% \ampere
%% \milliampere		oppure \mA
%% \microfarad		oppure \uF
%% \nanofarad			oppure \nF
%% \picofarad			oppure \pF
%% \kohm
%% \Mohm
%% \watt
%% \mW
%% \Hz
%% \kg
%% \joule
%% \mm
%% \mmq						(millimetri quadri -- mm²)
%% \tesla
%% \weber
%% \second
%% \coulomb



