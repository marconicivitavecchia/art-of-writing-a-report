%% Comando per indicare il prodotto con il punto centrale
%% Inserito cin mip.sty
%% Sintassi: A*B --> A·B							
%\mathcode`\*="8000
%{\catcode`\*\active\gdef*{\!\cdot\!}}

%%%%%%%%%%%%%%%%%%%%%%%%%%%%%%%%%%%%%%%%%%%%%%%%%%%%%%%%%%%%%%%%
%% comando per fare una serie di trattini che riempono la pagina
%% orizzontalmente
\def\dashfill{\hbox to \hsize{\cleaders\hbox{-}\hfill\hfil}}
%%%%%%%%%%%%%%%%%%%%%%%%%%%%%%%%%%%%%%%%%%%%%%%%%%%%%%%%%%%%%%%%

%%%%%%%%%%%%%%%%%%%%%%%%%%%%%%%%%%%%%%%%%%%%%%%%%%%%%%%%%%%%%%%%
%% Per comporre un vettore in bold. Sintassi \vect{v_1}
\DeclareRobustCommand{\vect}[1]{
  \ifcat#1\relax
    \boldsymbol{#1}
  \else
    \mathrm{\textbf{#1}}
  \fi}
%%%%%%%%%%%%%%%%%%%%%%%%%%%%%%%%%%%%%%%%%%%%%%%%%%%%%%%%%%%%%%%%

%%%%%%%%%%%%%%%%%%%%%%%%%%%%%%%%%%%%%%%%
%% Centra il contenuto di una cella di una tabella
%% Sintassi: \cell{<testo_cella>}{<separatore_colonna>}
%% dove <separatore_colonna> è per esempio |
\newcommand{\ccell}[2]{\multicolumn{1}{c#2}{\textbf{#1}}}
%%%%%%%%%%%%%%%%%%%%%%%%%%%%%%%%%%%%%%%%

%%%%%%%%%%%%%%%%%%%%%%%%%%%%%%%%%%%%%%%%
%% Per inserire figure in una signola colonna si
%% può utilizzare l'ambiente \begin{Figure} definito come segue.
%% Uso (occorre il package caption):
%% \begin{Figure}
%%		\centering
%%    \includegraphics[width=\textwidth]{figure/braccio_meccanico_small.png}%
%%    \captionof{figure}{figura 1}
%%    \label{fig:uno}
%% \end{Figure}
\newenvironment{Figure}
  {\par\medskip\noindent\minipage{\linewidth}}
  {\endminipage\par\medskip}
%%%%%%%%%%%%%%%%%%%%%%%%%%%%%%%%%%%%%%%%%
%% Oppure, caricando il package float:
%% \begin{figure}[H]%[htb!] %option=htb
%%		\centering
%%		\includegraphics[width=0.3\textwidth]{figure/braccio_meccanico_small.png}%
%%		\caption{figura 1}
%%		\label{fig:uno}
%% \end{figure}
%%%%%%%%%%%%%%%%%%%%%%%%%%%%%%%%%%%%%%%%%

\newcommand{\copyleft}{\reflectbox{\copyright}}

%%%%%%%%%%%%%%%%%%%%%%%%%%%%%%%%%%%%%%%%%%%%%%%%%%%%%%%%%%%%%%%%%
%% comando per fare una serie di trattini che riempono la pagina
%% orizzontalmente
\def\dashfill{\hbox to \hsize{\cleaders\hbox{-}\hfill\hfil}}
%%%%%%%%%%%%%%%%%%%%%%%%%%%%%%%%%%%%%%%%%%%%%%%%%%%%%%%%%%%%%%%%%

%%%%%%%%%%%%%%%%%%%%%%%%%%%%%%%%%%%%%%%%%%%%%%%%%%%%%%%%%%%%%%%%%
%% Comando per comporre una figura o una tabella
%% sull'intera larghezza della pagina
%% Parametri
%%	#1	posizione:	htbp!
%% 	#2	larghezza:	0 - 1
%%	#3	path e nome file (p.e. figure/abc.pdf)
%%	#4	testo del caption
%%	#5	nome label
%%	SINTASSI: \figtwocol{htbp!}{<num_tra_0_e_1>}{<path>/<nome_file>}{<caption>}{<label>}
\newcommand{\figtwocol}[5]{%
\end{multicols}
%%
\begin{figure*}[#1] %option=htb
\centering
    \includegraphics[width=#2\textwidth]{#3}%
    \caption{#4}\label{#5}
\end{figure*}
%%
\begin{multicols}{2}
}
%%%%%%%%%%%%%%%%%%%%%%%%%%%%%%%%%%%%%%%%%
\newcommand{\segue}{\quad\Rightarrow\quad}
%%%%%%%%%%%%%%%%%%%%%%%%%%%%%%%%%%%%%%%%%
\newcommand{\hlight}[1]{\textcolor{Maroon}{\sffamily#1}}
\newcommand{\hlightit}[1]{\textcolor{Maroon}{\sffamily\textit{#1}}}
\newcommand{\maroon}[1]{\textcolor{Maroon}{#1}}
%%%%%%%%%%%%%%%%%%%%%%%%%%%%%%%%%%%%%%%%%
%% Per cambiare al volo i margini di un paragrafo
%% Sintassi: 
%%             \begin{changemargin}{rs}{rd} 
%%                paragrafo da far rietrare
%%             \end{changemargin}
%% dove rs e rd sono i rientri destro e sinistro
\def\changemargin#1#2{\list{}{\rightmargin#2\leftmargin#1}\item[]}
\let\endchangemargin=\endlist 



