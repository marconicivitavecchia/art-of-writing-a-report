%% I package che seguono vengono caricati automaticamente dalla classe suftesi
%% calc, caption, enumitem, emptypage, epigraph, fancyhdr, fontenc, footmisc, geometry,
%% ifluatex, ifxetex, iwona, mathpazo, metalogo, microtype, mparhack, multicol, textcase,
%% titlesec, titletoc, varioref

%caption, color, crop, enumitem, emptypage, extramarks, fancyhdr, fixltxhyph,
%fontenc, geometry, ifxetex, ifluatex, ifthen, microtype, multicol, textcase,
%titlesec, titletoc, xkeyval; substitutefont and fontenc (pdfLATEXonly); lmodern
%(defaultfont=standard); textcomp, newpxtext, biolinum, inconsolata, newpxmath,
%mathalpha (defaultfont=palatino); textcomp, libertine, biolinum, inconsolata,
%newtxmath, mathalpha (defaultfont=libertine); textcomp, cochineal, biolinum,
%inconsolata, newtxmath, mathalpha (defaultfont=cochineal); mathpazo,
%beramono (defaultfont=compatibility).

% quando suftesi viene caricata con l'opzione defaultfont di deafult (cochineal)
% oppure libertine, viene caricato il package newtxmath che crea incompatibilità,
% se caricato a parte con il comando \usepackage{amsthm},
% con il package amsthm per il comando \openbox già definito in newtxmath alla linea
% 1772 (\DeclareRobustCommand{\openbox}{}).
% Analogamente accade se suftesi viene caricata con l'opzione defaultfont=palatino. In
% questo caso l'incompatibilità è nel package newpxmath dove alla linea 1503 viene
% dichiarato il comando: \DeclareRobustCommand{\openbox}{}.
% per risulvere momentaneamente il problema, caricare suftesi con l'opzione
% defaultfont=compatibility

%\usepackage{minitoc}
\usepackage[italian]{babel}
\usepackage[T1]{fontenc}
%\usepackage{ucs}
\usepackage[utf8]{inputenc}
\usepackage[babel,italian=guillemets]{csquotes}
\usepackage{makeidx}
	\makeindex
%\usepackage[babel]{csquotes}
\usepackage[backend=bibtex,backref,style=alphabetic,indexing]{biblatex}
%	\bibliography{bibliografia/biblio}
%% geometry: da non usare con la classe suftesi
%\usepackage[
%top=3cm,
%bottom=3.2cm,
%left=3cm, right=3cm,
%paper=a4paper,
%centering,
%pdftex,
%vcentering,
%marginratio=1:1]{geometry}
%\geometry{paper=a4paper,top=3cm,bottom=3cm}
\geometry{
%  top=15mm,
	includehead,
  includefoot,
%  headsep=2cm,
%  footskip=23mm,
  heightrounded,
}
%\usepackage[svgnames]{xcolor}
\usepackage{autobreak} % serve per andare acapo con le equazioni
	\allowdisplaybreaks
	
\usepackage{tabto}  % serve per le tabulazioni senza utilizzare l'ambiente "tabbing"

%%	Inserito in mip.sty
%%\usepackage{xspace} % serve per evitare che le parole che seguono nuovi comandi testuali
										% siano attaccate al testo definito da queli stessi comandi
										% Per esempio: \newcommand{\tred}{\textsc{3d}} senza nulla produce
										% quando invocato: "...bla bla bla \tredbla bla bla"
										% il comando \newcommand{\tred}{\textsc{3d}\xspace} compone il testo:
										% "bla bla bla \tred bla bla ..."

\usepackage[processing,arduino]{maker}  % serve per introdurre codice C con le caratteristiche
                    % tipiche del codice di Arduino
                    
\usepackage{cancel}	% permette di porre una barra per le sempleficazioni delle equazioni

%% package inserito in theimpress.sty e mip.sty in data 10/04/2018
%%\usepackage{icomma} % serve a evitare che i numeri dopo virgola siano distanziati
                    % viene usato al posto del comando \num{<n,m>} dove n,m è un
                    % numero decimale in cui n è la parte intera e m quella frazionaria.
										%	Se in modalità matematica si vogliono distanziare lettere e numeri
										%	basta inserire uno spazio. Per esempio: f(x, y)
										
\usepackage{circuitikz} % per fare circuiti elettrici ed elettronici
																
\usepackage{calc}		%	Permette di fare calcoli. Usato prevalentemente per scrivere
										%	nuovi comandi
\usepackage{float}	% Permette di definire o ridefinire gli stili degli oggetti
										% flottanti
\usepackage{fontawesome}
\usepackage{mip} 	%% package MarconiInstitutePress.
\usepackage{tabularx} %% estensione del package booktabs. Permette di fare celle che permettono
                      %% di andare automaticamente a capo, allineamento verticale ecc
\usepackage{diffcoeff} %% per scrivere facilemte equazioni differenziali, parziali ecc
\usepackage{witharrows}
\usepackage{afterpage} %% serve per fare il flush delle immagini che senza continuerebbero
                       %% ad essere spinte pagina dopo pagina, sezione dopo sezione ad 
                       %% essere spinte verso le ultime pagine. Sintassi:
                       %% \afterpage{\clearpage}
                       
\usepackage{colortbl}  %% Per colorare righe e colonne di tabelle
                       
%\usepackage[inline]{showlabels}	% opzione "right" per vedere le label,
																%"final" nella stesura finale
																
\usepackage{enumitem}
\usepackage{lscape}   %% per ruotare tabelle e figure

%%%%%%%%%%%%%%%%%%%%%%%%%%%%%%%%%%%%
\usepackage{draftwatermark}
\SetWatermarkText{Draft}
\SetWatermarkFontSize{8cm}
% Il comando che segue e' da usare unitamente al package draftwatermark
\SetWatermarkLightness{0.95}
