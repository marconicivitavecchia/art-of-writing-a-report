\documentclass[a4paper,10pt]{article}
%\documentclass[10pt]{standalone}

%\usepackage{ucs}
\usepackage[utf8]{inputenc}
\usepackage{amsmath}
\usepackage{amsfonts}
\usepackage{amssymb}
\usepackage{amsthm}
\usepackage{mathtools}
\usepackage[decimalsymbol=comma,alsoload=binary]{siunitx}
\usepackage{nccmath,textcomp}
\usepackage{diffcoeff}
\usepackage{subfigure}
\usepackage[italian]{babel}
\usepackage[OT1,T1]{fontenc}
\usepackage{graphicx}
\usepackage[colorlinks=true,linkcolor=Maroon,citecolor=blue,%
												draft=false,filecolor=red,urlcolor=Maroon]{hyperref}
\usepackage{autobreak} % serve per andare a capo con le equazioni
\allowdisplaybreaks
\usepackage{tabto}  % serve per le tabulazioni senza utilizzare l'ambiente "tabbing"
\usepackage{xspace} % serve per evitare che le parole che seguono il comando \tred
                    % siano attaccate con il testo definito dallo stesso comando
\usepackage{cancel}
\usepackage{icomma} % serve a evitare che i numeri dopo virgola siano distanziati
                    % viene usato al posto del comando \num{<n,m>} dove n,m è un
                    % numero decimale in cui n è la parte intera e m quella frazionaria.
										%	Se in modalità matematica si vogliono distanziare lettere e numeri
										%	basta inserire uno spazio. Per esempio: f(x, y)
\usepackage{enumitem}
\usepackage[svgnames]{xcolor}
%\RequirePackage{draftwatermark}
%\SetWatermarkText{Confidenziale}
%\SetWatermarkFontSize{4cm}
% Il comando che segue e' da usare unitamente al package draftwatermark
%\SetWatermarkLightness{0.8}
\usepackage{relsize}
\usepackage{booktabs,array,rotating}
\usepackage{tabularx}
\usepackage{multirow}
\usepackage{colortbl}

\newcommand{\volt}{\si{\,\volt}}
\newcommand{\ampere}{\si{\,\ampere}}
\newcommand{\ohm}{\si{\,\ohm}}



\date{}

\begin{document}


\begin{tabular}{>{\itshape}c >{\itshape}c >{\itshape}c >{\itshape}c >{\itshape}c >{\itshape}c}
%%
\toprule
\textbf{Colonna indice} &  & \multicolumn{4}{c}{\textbf{Colonne dati}} \\
\midrule
%%
$\overbrace{\text{\hspace{25mm}}}$ & \multicolumn{5}{c}{$\overbrace{\text{\hspace{60mm}}}$} \\
entrata 1 &  & cella 1.1 & cella 1.2 & $\cdots$ &  cella 1.m \\
entrata 2 &  & cella 2.1 & cella 2.2 & $\cdots$ &  cella 2.m \\
$\vdots$  &  & $\vdots$  & $\vdots$ & $\vdots$ &  $\vdots$ \\
entrata n &  & cella 1.m  & cella n.2 & $\cdots$ & cella n.m \\
\bottomrule
\end{tabular}
%%%%%%%%%%%%%%%%%%%%%%%%%%%%%%%%%%%%%%%%%%%%%%%%%%%%%%%%%%%%%%%%%%%%%%%%%%
%%%%%%%%%%%%%%%%%%%%%%%%%%%%%%%%%%%%%%%%%%%%%%%%%%%%%%%%%%%%%%%%%%%%%%%%%%

\centering
\vspace{10mm}
\begin{tabularx}{0.7\textwidth}{ccccc>{\centering\arraybackslash}X}
%%
\toprule
               & \multicolumn{2}{c}{\textit{Voltometro}} & \multicolumn{2}{c}{\textit{Amperometro}} & \\
               \cmidrule(lr){2-3}\cmidrule(lr){4-5}
\textbf{n°}    & div.  & $V_m$     & div.  & $I_m$        & $R_m=\frac{V_m}{I_m}$ \\
\textbf{prova} & lette & $(\volt)$ & lette & $(\ampere)$  &      ($\ohm$)         \\
\midrule
%%
1   			& $141,5$ 	&	$14,15$ 	&	$69,7$		& $0,349$ 	& $40,54$	 \\
%%
2   			& $136,9$ 	&	$13,69$ 	&	$67,0$ 		&	$0,335$	 	& $40,87$ 	\\
%%
$\vdots$  &	$\vdots$	& $\vdots$	& $\vdots$	&	$\vdots$	&	$\vdots$  \\
%%
n   			& $101,5$		&	$10,15$ 	& $49,9$		& $0,249$		& $40,76$		\\
\bottomrule
\end{tabularx}

%%%%%%%%%%%%%%%%%%%%%%%%%%%%%%%%%%%%%%%%%%%%%%%%%%%%%%%%%%%%%%%%%%%%%%%%%%
%%%%%%%%%%%%%%%%%%%%%%%%%%%%%%%%%%%%%%%%%%%%%%%%%%%%%%%%%%%%%%%%%%%%%%%%%%
\vspace{10mm}
\begin{tabularx}{0.7\textwidth}{ccccc>{\centering\arraybackslash}X}
%%
\toprule
               & \multicolumn{2}{c}{\textit{Voltometro}} & \multicolumn{2}{c}{\textit{Amperometro}} & \\
               \cmidrule(lr){2-3}\cmidrule(lr){4-5}
\textbf{n°}    & div.  & $V_m$     & div.  & $I_m$        & $R_m=\frac{V_m}{I_m}$ \\
\textbf{prova} & lette & $(\volt)$ & lette & $(\ampere)$  &      ($\ohm$)         \\
\midrule
%%
1   			& $141,5$ 	&	$14,15$ 	&	$69,7$		& $0,349$ 	& $40,54$	 \\
%%
\rowcolor[gray]{0.9}
2   			& $136,9$ 	&	$13,69$ 	&	$67,0$ 		&	$0,335$	 	& $40,87$ 	\\
%%
$\vdots$  &	$\vdots$	& $\vdots$	& $\vdots$	&	$\vdots$	&	$\vdots$  \\
%%
\rowcolor[gray]{0.9}
n   			& $101,5$		&	$10,15$ 	& $49,9$		& $0,249$		& $40,76$		\\
\bottomrule
\end{tabularx}
%%%%%%%%%%%%%%%%%%%%%%%%%%%%%%%%%%%%%%%%%%%%%%%%%%%%%%%%%%%%%%%%%%%%%%%%%%
\vspace{10mm}
%%%%%%%%%%%%%%%%%%%%%%%%%%%%%%%%%%%%%%%%%%%%%%%%%%%%%%%%%%%%%%%%%%%%%%%%%%
\vspace{10mm}
\begin{tabularx}{0.9\textwidth}{lXX}
\toprule
Periodo & Fenomeni geologici & Biosfera \\
\midrule
\textbf{Giurassico} & Periodo caratterizzato da variazioni del
livello del mare; prevalenza delle terre emerse in America, Asia,
Australia. & Fauna: compaiono i primi marsupiali; dominano i grandi
rettili (dinosauri). Flora: predominano le conifere. \\
\midrule
\textbf{Triassico} & Intensa l’erosione dei continenti; profonde
fratture da cui escono lave che originano altopiani estesi.
& Fauna: si diffondono i rettili; nei mari prosperano pesci e
invertebrati. Flora: si sviluppano alghe caratteristiche. \\
\bottomrule
\end{tabularx}

%%%%%%%%%%%%%%%%%%%%%%%%%%%%%%%%%%%%%%%%%%%%%%%%%%%%%%%%%%%%%%%%%%%%%%%%%%
%%%%%%%%%%%%%%%%%%%%%%%%%%%%%%%%%%%%%%%%%%%%%%%%%%%%%%%%%%%%%%%%%%%%%%%%%%

\vspace{10mm}
\begin{tabular}{|c|c|c|c|c|}
%%
\hline
\textbf{Colonna 1} & \textbf{Colonna 2} & \textbf{Colonna 3} & $\cdots$ & \textbf{Colonna m}\\
\hline
%%
dato 1.1 & dato 1.2 & dato 1.3 & $\cdots$ &  dato 1.m \\
\hline
dato 2.1 & dato 2.2 & dato 2.3 & $\cdots$ &  dato 2.m \\
\hline
$\vdots$  & $\vdots$  & $\vdots$ & $\vdots$ &  $\vdots$ \\
\hline
dato n.1 & dato n.2 & dato n.3 & $\cdots$ &  dato n.m \\
\hline
\end{tabular}
%%%%%%%%%%%%%%%%%%%%%%%%%%%%%%%%%%%%%%%%%%%%%%%%%%%%%%%%%%%%%%%%%%%%%%%%%%
%%%%%%%%%%%%%%%%%%%%%%%%%%%%%%%%%%%%%%%%%%%%%%%%%%%%%%%%%%%%%%%%%%%%%%%%%%

\vspace{10mm}
\begin{tabular}{ccccc}
%%
\toprule
\textbf{Colonna 1} & \textbf{Colonna 2} & \textbf{Colonna 3} & $\cdots$ & \textbf{Colonna m}\\
\midrule
%%
dato 1.1 & dato 1.2 & dato 1.3 & $\cdots$ &  dato 1.m \\
%%
dato 2.1 & dato 2.2 & dato 2.3 & $\cdots$ &  dato 2.m \\
%%
$\vdots$  & $\vdots$  & $\vdots$ & $\vdots$ &  $\vdots$ \\
%%
dato n.1 & dato n.2 & dato n.3 & $\cdots$ &  dato n.m \\
\bottomrule
\end{tabular}

\end{document}
